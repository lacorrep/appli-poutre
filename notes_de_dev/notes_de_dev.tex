\documentclass[draft]{report}

% Package personnalisé
\usepackage{myPackage}

\title{Appli Poutre - Notes de développement}
\author{Utilisateur GitHub : \url{selakant}}
\date{Juin 2017}

\begin{document}

\maketitle

\tableofcontents


%%%%%%%%%%%%%%%%%%%%%%%%%%%%%%%%%%%%%%%%%%%%%%%
\chapter{Modélisation physique}

\section{Hypothèses}
\begin{itemize}
\item Poutre isostatique (vérifié avant chaque résolution)
\item Petites déformations (la déformation n'induit pas de nouveaux efforts internes)
\item Sections droites (indéformables)
\item Poutres d'Euler-Bernoulli : les sections restent perpendiculaires à la ligne moyenne
\end{itemize}

L'hypothèse d'Euler-Bernoulli implique que la rotation de la section est égale à la pente de la déformée (en petites transformations).

\section{Équations du problème}

Les équations à résoudre sont~:
\begin{systeme}
\der{u}{x} &=& \dfrac{1}{ES} N(x) \\
\\
\dfrac{\mathrm{d}^2 v}{\mathrm{d}x^2} &=& \dfrac{1}{EI} M(x) \\
\end{systeme}
On doit avoir 1 CL sur $u$ et 2 CL sur $v$.

Les deux problèmes sont clairement découplés.

\subsection{Calcul des efforts internes $N$ et $M$}

Les efforts sont calculés par coupure en isolant la partie droite.\todo{}

\subsection{Calcul des déplacements horizontaux $u$}
On intègre la première équation entre $0$ et $x$~:
\[ u(x) = u(0) + \int_0^x \dfrac{1}{ES} N(x) \intd x \]
Cependant $u(0)$ n'est pas forcément connu.

On pose
\[ U(x) = \int_0^x \dfrac{1}{ES} N(x) \intd x \]
ce qui est la solution lorsque $u(0) = 0$.

Via les données saisies par l'utilisateur, on connaît la position $x_0$ de la liaison qui empêche le déplacement horizontal. On impose alors de manière forte $u(x_0) = 0$~:

\important{ u(x) = U(x) - U(x_0) }

\subsection{Calcul des déplacements verticaux $v$}

\todo[inline]{à faire}
% WIP
Via les données saisies par l'utilisateur, on connaît les positions $x_i$ de la ou des liaisons qui empêchent le déplacement vertical. On impose alors cette condition de manière forte.

Si on a une seule liaison en $x_1$~: la solution est définie à une constante près
\important{ v(x) = V(x) - V(x_1) }

Si on a deux liaisons en $x_1$ et $x_2$~: la solution est définie à un polynôme près. On pose
\[ \alpha = \dfrac{ V(x_2)-V(x_1) }{ x_2-x_1 } \]
\important{ v(x) = V(x) - \Big( \alpha \times (x - x_1) + V(x_1) \Big) }

% FIN WIP





%%%%%%%%%%%%%%%%%%%%%%%%%%%%%%%%%%%%%%%%%%%%%%%
\chapter{Implémentation}

% Parler des unités utilisées : EI = 1, longueur de poutre = 1., efforts...

\section{Affichage de la poutre}

\todo{parler de la marge sur les côtés pour afficher la déformée}

\section{Discrétisation des équations d'équilibre local}

On prend autant de points que la poutre fait de pixels à l'écran.

\section{Détermination des efforts d'appui}

On note $\vec R$, $M$ les réactions des liaisons et $\vec F$, $C$ les chargements. Le PFS donne~:

\begin{systeme}
\vec R + \vec F =& \vec 0 \\
M + C =& 0 \\
\end{systeme}
\begin{systeme}
R_X =& - F_X \\
R_Y =& - F_Y \\
M =& - C \\
\end{systeme}

L'unique CL sur $u$ est introduite par un unique effort horizontal $R_X$. Les deux CL sur $v$ et ses dérivées peuvent être appliquées par un effort vertical $R_Y$ et par un autre effort vertical $R_Y'$ ou un moment $M$.

\[ \Rightarrow
\begin{bmatrix}
\dots & \dots & \dots \\
\dots & \dots & \dots \\
\dots & \dots & \dots \\
\end{bmatrix}
\begin{bmatrix}
R_X \\
R_Y \\
R_Y' \text{ ou } M
\end{bmatrix}
=
\begin{bmatrix}
- \sum F_X \\
- \sum F_Y \\
- \sum C \\
\end{bmatrix} \]

Les efforts horizontaux ne génèrent pas de moments sur la poutre. Ainsi l'équilibre horizontal ne dépend que des efforts horizontaux. Il n'y a qu'une seule force de réaction horizontale.
\important{ R_X = - \sum F_X }

Pour déterminer les autres efforts, on résout le système suivant par la méthode de Cramer~:
\important{ \begin{bmatrix}
\dots & \dots \\
\dots & \dots \\
\end{bmatrix}
\begin{bmatrix}
R_Y \\
R_Y' \text{ ou } M
\end{bmatrix}
=
\begin{bmatrix}
- \sum F_Y \\
- \sum C \\
\end{bmatrix}  }

Sur la première ligne, $j$-ième colonne~: on a $1$ si la $j$-ième inconnue est une force, $0$ si c'est un moment.

Sur la deuxième ligne, $j$-ième colonne~: on a $x$ la distance de la force à l'extrémité gauche de la poutre si la $j$-ième inconnue est une force, $1$ si c'est un moment.

\paragraph{Exemples} pour une poutre bi-appuyée (en $x$ et $x'$)~:
\[ \begin{bmatrix}
\red 1 & \blue 1 \\
\red x & \blue{x'} \\
\end{bmatrix}
\begin{bmatrix}
\red{R_Y} \\
\blue{R_Y'}
\end{bmatrix}\]
pour une poutre encastrée (en $x$)~:
\[ \begin{bmatrix}
\red 1 & \blue 0 \\
\red x & \blue 1 \\
\end{bmatrix} 
\begin{bmatrix}
\red{R_Y} \\
\blue M
\end{bmatrix}\]

\paragraph{Résolution}

\important{
[A] \{R\} = \{F\}
\Leftrightarrow
 \begin{bmatrix}
a_{00} & a_{01} \\
a_{10} &a_{11} \\
\end{bmatrix}
\begin{bmatrix}
R1 \\
R2
\end{bmatrix}
=
\begin{bmatrix}
- \sum F_Y \\
- \sum C \\
\end{bmatrix} 
}

\[ R1 = \dfrac{ a_{01} \times \sum C - a_{11} \times \sum F_Y }{\det A} \]
\[ R2 = \dfrac{ a_{10} \times \sum F_Y - a_{00} \times \sum C }{\det A} \]



%%%%%%%%%%%%%%%%%%%%%%%%%%%%%%%%%%%%%%%%%%%%%%%
\chapter{Bibliothèques et dépendances}

\noindent
\begin{tabularx}{\textwidth}{llXl}
\hline
Nom & Version & Utilité & Source \\
\hline\\
jQuery & 3.2.1 & Permet de manipuler facilement le DOM. & \url{jquery.com} \\
jQuery UI & 1.12.1 & Permet de définir rapidement des composants d'interface (\textit{drag and drop}, redimensionnement). & \url{jqueryui.com} \\
Touch Punch & 0.2.3 & Assure la compatibilité de jQuery UI avec les événements tactiles. & \url{touchpunch.furf.com} \\
Bootstrap & 3.3.7 & Framework d'interface et responsive design. & \url{getbootstrap.com} \\
\end{tabularx}

\end{document}